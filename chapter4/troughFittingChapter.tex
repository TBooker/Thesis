\chapter{Estiamting parameters of selective sweeps from patterns of genetic diversity in house mice}
\chaptermark{Trough fitting}



%%%%%%%%%%%%%%%%%%%%%%%%%%%%%%%%%%%%%%%%%%%%%%%%%%%%
%
%  #   #    #  #####  #####  ######
%  #   ##   #    #    #   #  #    #     
%  #   # #  #    #    #####  #    #
%  #   #  # #    #    #  #   #    #           
%  #   #   ##    #    #   #  #    #  
%  #   #    #    #    #   #  ######    
%
%%%%%%%%%%%%%%%%%%%%%%%%%%%%%%%%%%%%%%%%%%%%%%%%%%%%

\section*{Introduction}

In the past 30 years of population genetic research it has become clear that natural selection shapes patterns of nucleotide diversity across the genomes of many species \citep{RN154, RN117}. Because genetically linked sites do not evolve independently, selection acting at one site may have consequences for another. The consequences of selection at linked sites are intrinsically linked to the frequency and strength of selected mutations as well as, crucially, the rate of recombination (REF DUMP). Two main modes of selection at linked sites have been identified; selective sweeps caused by the spread of advantageous mutations and background selection caused by the removal of deleterious variants. The two processes are related and can both potentially explain the positive correlations between nucleotide diversity and recombination rate reported in many species \citep{RN117}. However, the proportion of nonsynyonomous substitutions attributable to adaptive evolution ($\alpha$) is typically high (50\%) (\citealt{RN215}; but see \citealt{RN352} for caveats), suggesting that selective sweeps may play a substantial role in shaping nucleotide diversity across the genomes of many species.


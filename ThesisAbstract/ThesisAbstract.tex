% Thesis abstract

\chapter*{ThesisAbstract}
\chaptermark{Thesis Abstract}

It is well understood that nucleotide diversity varies across the genomes of many eukaryotic species in ways consistent with the effects of natural selection. However, the contribution of selection on advantageous and deleterious mutations to the observed variation is less well understood. In this thesis, I aim to disentangle the contribution of background selection and selective sweeps to patterns of genetic diversity in the mouse genome, thus furthering our understanding of natural selection in mammals. In chapter 1, I introduce core concepts in evolutionary genetics and describe how recombination and selection interact to shape patterns of genetic diversity. I will then describe three projects in which I examine aspects of molecular evolution in house mice. In the first of these, I estimate the landscape of recombination rate variation in wild mice using population genomic data. In the second, I estimate the distribution of fitness effects for new mutations, based on the site frequency spectrum, I then analyze population genomic simulations parametrized using my estimates. In the third, I use a model of selective sweeps to estimate and compare the strength of selection occurring in protein-coding and regulatory regions of the mouse genome. This thesis demonstrates that selective sweeps, are responsible for a large amount of the variation in genetic diversity across the mouse genome.
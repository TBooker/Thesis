
Included here are the supplementary figures and tables for Chapter 2 as well as a reproduction of \cite{RN340}.

\section{Supplementary Material}

\linespread{1}
\begin{table}[h!]
\centering
\caption[Comparison of the fit of demographic models based on the analysis of 4-fold sites and CNE-flanks in \textit{M. m. castaneus}]{Comparison of the fit of demographic models based on the analysis of 4-fold sites and CNE-flanks in \textit{M. m. castaneus}.}
 \begin{tabular}{c c c c c } 

\toprule
       &Epochs&$\Delta lnL$& $\chi^2$& \# Estimated Parameters \\ \hline
\multirow{3}{*}{4-fold} & 	1 &	1,620	 &	22,500 	&	2 \\
		 &	2 &	159		 &	2,930 	&	4 \\
 		&	3 &	0.0		 &	553 		&	6 \\ \hdashline
\multirow{3}{*}{CNE-flank} &	1 &	19,100	 &	53,500 	&	2 \\
		 &	2 &	1,350	 &	5,070	&	4 \\
 		&	3 &	0.0 		 &	975 		&	6 \\
\bottomrule
\end{tabular}
\label{tab:CS1}
\end{table}

\begin{table}[h!]
\centering
\caption[Parameters of the best-fitting demographic model estimated from the analysis of 4-fold and CNE-flanking sites]{Parameters of the best-fitting demographic model estimated from the analysis of 4-fold and CNE-flanking sites. }
 \begin{tabular}{c c c c c } 

\toprule
	&4-fold	&CNE-flank \\ \hline
N2/N1&	0.40&	0.07 \\
t2/N1&	0.44&	0.17 \\
N3/N1&	0.40&	1.00 \\
t3/N1&	1.10&	0.63 \\
\bottomrule

\end{tabular}
\label{tab:C3S2}
\end{table}
\begin{table}[h!]
\centering
\caption[Parameters of the 3-epoch demographic model at different sample sizes]{Parameters of the 3-epoch demographic model at different sample sizes. Down sampled datasets were generated by randomly selecting alleles, with respect to frequency, from the full dataset of 10 individuals.}
 \begin{tabular}{c c c c c } 

\toprule
\multirow{2}{*}{Parameter} & \multicolumn{3}{c}{Number of alleles sampled} \\ 
	&	$n = 10$ & 	$n = 16$ & 	$n = 20$ \\ \hline
N2/N1 &	0.030 &	0.030 &	0.060 \\
t2/N1 &	0.204 &	0.140 &	0.181 \\
N3/N1 &	0.120 &	0.200 &	0.800 \\
t3/N1 &	0.080 &	0.220 &	0.461 \\
\bottomrule

\end{tabular}
\label{tab:C3S3}
\end{table}
\begin{table}[h!]
\centering
\caption[Likelihood differences between models of the dDFE fitted with or without a single class of adaptive mutations]{Likelihood differences between models of the deleterious DFE (dDFE) fitted with or without a single class of adaptive mutations.}
 \begin{tabular}{c c c c } 

\toprule
\multirow{2}{*}{\textbf{Site Type}} 	& \multirow{2}{*}{\textbf{dDFE Model}}& 	\multicolumn{2}{c}{\textbf{$\Delta lnL$}} \\
    &  &   \textbf{dDFE}	 & \textbf{dDFE + Adaptive Mutations} \\ \hline
\multirow{2}{*}{\textbf{0-fold}} &	1-Class	&49,300	&4.18 \\
	   &2-Class	&129 	&0.00 \\
	   &3-Class	&129	    &0.00 \\
	   &Gamma	&247	 	&4.18 \\ \hdashline
\multirow{2}{*}{\textbf{CNE}}	   &1-Class	&51,000	&245 \\
	   &2-Class	&1,660	&3.41 \\
	   &3-Class	&1,480	&0.00 \\
	   &Gamma	&2,310	&19.3 \\ \hdashline
\multirow{2}{*}{\textbf{UTR}}	   &1-Class	&6,170	&32.7 \\
	   &2-Class	&335	    &0.00 \\
	   &3-Class	&335	    &0.00 \\
	   &Gamma	&970	    &13.5 \\
\bottomrule

\end{tabular}
\label{tab:C3S4}
\end{table}
\pagebreak 
 
 \begin{figure}
   \centering      
   \noindent\makebox[\textwidth]{\includegraphics[width=\textwidth]{/Users/s0784966/Dropbox/Thesis/chapter2Appendix/Figures/FigureS1.pdf}}
 \caption[The effect of block penalty on recombination rate inference]{The effect of switch errors and block penalty on the mean recombination rate inferred using LDhelmet. Block penalties (b) of 10, 25, 50 and 100 were used, shown in the vertically ordered facets from top to bottom.}
 \label{fig:C2SF1}
\end{figure}

 
 \begin{figure}
   \centering      
   \noindent\makebox[\textwidth]{\includegraphics[width=\textwidth]{/Users/s0784966/Dropbox/Thesis/chapter2Appendix/Figures/FigureS2.pdf}}
 \caption[Recombination rates maps for each mouse chromosome]{Comparison of recombination rates inferred for \textit{M. m. castaneus} using LDhelmet
 and recombination rates reported by \cite{RN232}. Recombination rates in units of $\rho / bp$
 for the \textit{castaneus} map were converted to cM/Mb by scaling using the genetic length of the
 corresponding chromosome in the Cox map.}
 \label{fig:C2SF2}
\end{figure}


 \begin{figure}
   \centering      
   \noindent\makebox[\textwidth]{\includegraphics[width=\textwidth]{/Users/s0784966/Dropbox/Thesis/chapter2Appendix/Figures/FigureS3.pdf}}
 \caption[Correlations between recombination maps for each chromosome]{Pearson correlation coefficients between the recombination map inferred for \textit{M. m. castaneus}, the \cite{RN156} map and the \cite{RN232} map for each chromosome separately. Comparisons for the X-chromosome were not made with Brunschwig et al. (2012) map, as it was not included in that study.}
 \label{fig:C2SF3}
\end{figure}
 
 \begin{figure}
   \centering      
   \noindent\makebox[\textwidth]{\includegraphics[width=\textwidth]{/Users/s0784966/Dropbox/Thesis/chapter2Appendix/Figures/FigureS4.pdf}}
 \caption[A snapshot of the recombination landscape for genomic region containing the PRDM9 gene]{A snapshot of the recombination rate landscape inferred for \textit{M. m. castaneus} with LDhelmet block penalties. Recombination hotspots were inferred using the map constructed using a block penalty of 100.}
 \label{fig:C2SF4}
\end{figure}
 
 
 \begin{figure}
   \centering      
   \noindent\makebox[\textwidth]{\includegraphics[width=\textwidth]{/Users/s0784966/Dropbox/Thesis/chapter2Appendix/Figures/FigureS5.pdf}}
 \caption[Lorenz curves for each of the chromosomes of \textit{M. m. castaneus}]{Lorenz curves for each of the autosomes of \textit{M. m. castaneus} as well as the X-chromosome.}
 \label{fig:C2SF5}
\end{figure}
 
 
 \begin{figure}
   \centering      
   \noindent\makebox[\textwidth]{\includegraphics[width=\textwidth]{/Users/s0784966/Dropbox/Thesis/chapter2Appendix/Figures/FigureS6.pdf}}
 \caption[The effect of switch errors on recombination rate inference]{The effect of switch errors on the mean recombination rate inferred using LDhelmet with a block penalty of 100. Each black point represents results for a window of 4000 SNPs, with 200 SNPs overlapping between adjacent windows, using sequences simulated in SLiM for a constant value of $\rho/bp$. Red points are mean values. Switch errors were randomly incorporated at heterozygous SNPs with probability 0.0046. The dotted line shows the value when the inferred and true rates are equal}
 \label{fig:C2SF6}
\end{figure}



\pagebreak
\section{Booker \emph{et al.} 2017 - Genetics}
\includepdf[pages=-, scale = 0.8 , pagecommand={\pagestyle{fancy}}]{/Users/s0784966/Dropbox/Thesis/chapter2Appendix/Booker_et_al2017_Genetics.pdf}
\linespread{2}
